% !Mode:: "TeX:UTF-8"
\documentclass{../common/tufte-latex/tufte-handout}

\title{Migrating SVN projects to Git with git-svn}

\author{S\'ebastien Dawans}

\date{07 February 2014} % without \date command, current date is supplied

%\geometry{showframe} % display margins for debugging page layout
\usepackage[utf8]{inputenc}
\usepackage{graphicx} % allow embedded images
  \setkeys{Gin}{width=\linewidth,totalheight=\textheight,keepaspectratio}
  \graphicspath{{graphics/}} % set of paths to search for images
\usepackage{amsmath}  % extended mathematics
\usepackage{booktabs} % book-quality tables
\usepackage{units}    % non-stacked fractions and better unit spacing
\usepackage{multicol} % multiple column layout facilities
\usepackage{lipsum}   % filler text
\usepackage{fancyvrb} % extended verbatim environments
  \fvset{fontsize=\normalsize}% default font size for fancy-verbatim environments
\usepackage{listings}
\lstset{showstringspaces=false}
\usepackage[usenames]{xcolor}
\usepackage{soul}
\usepackage{enumerate}

\lstdefinestyle{BashInputStyle}{
  language=bash,
  basicstyle=\footnotesize\ttfamily,
  %numbers=left,
  %numberstyle=\tiny,
  %numbersep=3pt,
  frame=tb,
  columns=fullflexible,
  backgroundcolor=\color{yellow!20},
  linewidth=0.95\linewidth,
  xleftmargin=0.05\linewidth,
  moredelim=**[is][\color{red}]{§}{§},
  moredelim=**[is][\color{OliveGreen}]{°}{°}
}

% Standardize command font styles and environments
\newcommand{\doccmd}[1]{\texttt{\textbackslash#1}}% command name -- adds backslash automatically
\newcommand{\docopt}[1]{\ensuremath{\langle}\textrm{\textit{#1}}\ensuremath{\rangle}}% optional command argument
\newcommand{\docarg}[1]{\textrm{\textit{#1}}}% (required) command argument
\newcommand{\docenv}[1]{\textsf{#1}}% environment name
\newcommand{\docpkg}[1]{\texttt{#1}}% package name
\newcommand{\doccls}[1]{\texttt{#1}}% document class name
\newcommand{\docclsopt}[1]{\texttt{#1}}% document class option name
\newenvironment{docspec}{\begin{quote}\noindent}{\end{quote}}% command specification environment

\begin{document}

\maketitle% this prints the handout title, author, and date

\begin{abstract}
\noindent
This handout is part of a series of practical courses on Git.
\marginnote{git-svn is distributed with the standard Unix Git client \url{https://www.kernel.org/pub/software/scm/git/docs/git-svn.html}}
A common issue when getting started with Git is forgetting old habits adopted while using SVN.
A great remedy for this is to stop using SVN altogether; that sometimes means having to migrate existing, long-term projects from SVN to Git.
Here, we will see how easy it is to migrate SVN projects to Git with the \texttt{git-svn} tool.
A migration can take as few as 10 minutes, so let's get started!
\end{abstract}

\section{What this handout is not}

This is \textit{not} a complete reference on how the git-svn tool works.
There is already plenty of documentation on that, a great one being the Pro Git book by Scott Chacon, available online \cite{git-scm-ch8}.
git-svn is a Git client offering features of Git where branching, merging, staging, committing, viewing and rewriting the history can all be done locally.
\marginnote{Rewriting non-pushed, or non "SVN-comitted" history, to be exact.}
Git fanatics stuck on pre-historic SVN-based projects may certainly find this useful.

This handout only scratches the surface of git-svn as an SVN proxy.
Instead, we will focus on leveraging this tool to migrate SVN projects to Git once and for all.

\section{Preparation}

\subsection{Installation}

For this tutorial, we will use a sample SVN project hosted on good ol' SourceForge. 
\marginnote{SourceForge definition of SVN: Enterprise-class centralized version control for the masses. Classy.}
That means you'll need both a Git client and an SVN client.
The Git installation is already covered in a previous handout.

To install SVN, Linux users should install the \textbf{subversion} package with

\begin{lstlisting}[style=BashInputStyle]
  $ apt-get install subversion
\end{lstlisting}

Windows users may download and execute the TortoiseSVN installer and verify that the svn command is properly added to the \texttt{PATH} variable.
\marginnote{TortoiseSVN \url{http://tortoisesvn.net/downloads.html}}

\subsection{\st{Clone} Checkout the SVN project}

Last but not least, let's execute our one and only SVN command to grab the SVN project from SourceForge.

\begin{lstlisting}[style=BashInputStyle]
  $ svn checkout svn://svn.code.sf.net/p/svngitmigration/code/ svngit-project
  $ cd svngit-project
\end{lstlisting}

\section{Analyzing the SVN project}

A typical SVN project contains a trunk with the latest development version of the application.
The trunk is an actual physical folder in the working tree, usually named \textbf{trunk}.
As opposed to Git, branches and tags have a physical location in the working tree as well, so a common practice is to store these in top-level \textbf{branches} and \textbf{tags} folders, next to the trunk.
Our sample project follows this convention:

%TODO: Update Tree listing once the SVN project is finished
\begin{lstlisting}[style=BashInputStyle]
  $ tree -L 2
  |-- branches
  |-- tags
  |-- trunk
      |-- README.txt
\end{lstlisting}

Let's check the log:

%TODO: complete the log when project is done
\begin{lstlisting}[style=BashInputStyle]
  $ svn log
\end{lstlisting}

Before we go on, let's consider some differences between SVN and Git and what this implies in terms of migrating to Git:

\begin{itemize}
 \item \textbf{SVN branches and tags have a physical location}. This is a major difference to Git, where branches and tags are very lightweight: mere labels and pointers to commits. Thus, there isn't a 1-to-1 match between an SVN working tree and a Git working tree.
 \item \textbf{Author names are not formatted in the same way}. SVN doesn't require any structure for author names, it's usually the committer's account username on the Subversion system. On the other hand, Git formats the author information with \texttt{"First Last <email@domain.com>"}. We need to provide a mapping for this.
 \item \textbf{SVN accepts blank commits, Git doesn't}. True, but don't panic. Git will allow the undisciplined SVN user to import his blank commits.
\end{itemize}

\bibliography{../common/refs}
\bibliographystyle{plainnat}

\end{document}
