% !Mode:: "TeX:UTF-8"
\documentclass{../common/tufte-latex/tufte-handout}

\title{Git hands-on, part II: Single User Operations}
\author{S\'ebastien Dawans}

%\date{07 February 2014} % without \date command, current date is supplied

%\geometry{showframe} % display margins for debugging page layout
\usepackage[utf8]{inputenc}
\usepackage{graphicx} % allow embedded images
  \setkeys{Gin}{width=\linewidth,totalheight=\textheight,keepaspectratio}
  \graphicspath{{graphics/}} % set of paths to search for images
\usepackage{amsmath}  % extended mathematics
\usepackage{booktabs} % book-quality tables
\usepackage{units}    % non-stacked fractions and better unit spacing
\usepackage{multicol} % multiple column layout facilities
\usepackage{lipsum}   % filler text
\usepackage{fancyvrb} % extended verbatim environments
  \fvset{fontsize=\normalsize}% default font size for fancy-verbatim environments
\usepackage{listings}
\lstset{showstringspaces=false}
\usepackage[usenames]{xcolor}
\usepackage{hyperref}

\lstdefinestyle{BashInputStyle}{
  language=bash,
  basicstyle=\footnotesize\ttfamily,
  %numbers=left,
  %numberstyle=\tiny,
  %numbersep=3pt,
  frame=tb,
  columns=fullflexible,
  backgroundcolor=\color{yellow!20},
  linewidth=0.95\linewidth,
  xleftmargin=0.05\linewidth,
  moredelim=**[is][\color{red}]{§}{§},
  moredelim=**[is][\color{OliveGreen}]{`}{`}
}

% Standardize command font styles and environments
\newcommand{\doccmd}[1]{\texttt{\textbackslash#1}}% command name -- adds backslash automatically
\newcommand{\docopt}[1]{\ensuremath{\langle}\textrm{\textit{#1}}\ensuremath{\rangle}}% optional command argument
\newcommand{\docarg}[1]{\textrm{\textit{#1}}}% (required) command argument
\newcommand{\docenv}[1]{\textsf{#1}}% environment name
\newcommand{\docpkg}[1]{\texttt{#1}}% package name
\newcommand{\doccls}[1]{\texttt{#1}}% document class name
\newcommand{\docclsopt}[1]{\texttt{#1}}% document class option name
\newenvironment{docspec}{\begin{quote}\noindent}{\end{quote}}% command specification environment

\begin{document}

\maketitle% this prints the handout title, author, and date

\begin{abstract}
\noindent
This handout covers basic Git commands in a first hands-on session, following the introductory presentation and handout (see \texttt{part I: Introduction \& Installation}).
The first part looks at how to ignore files from Git.
The rest documents the history modification commands that were introduced orally during the last session in answer to specific questions.
\end{abstract}

\section{New repository: init or clone?}

When getting started on a project as a developer, you will either initialize a git repository or clone an existing one.

\subsection{Cloning an existing repository}

If you are joining a project with an existing codebase and versioned with Git, you can simply issue a \texttt{git clone} command on the correct URL to create a local repository: \marginnote{Among the several options of the clone command, \texttt{directory} is probably the most useful one at this point: it allows to specify the name of the repository folder. The repository and working tree are fully contained within that folder, so any folder name may be given.}

\begin{lstlisting}[style=BashInputStyle]
  $ git clone <url> [<directory>]
\end{lstlisting}

This will create a folder in which the Git repository will be cloned.
To be exact, this folder will contain the \textbf{working tree}, which is a particular snapshot of the source code being tracked, and the \textbf{local repository} contaning Git-related metadata, inside a \texttt{.git} subfolder.

\subsection{Creating a new repository: empty clone}

If you are starting off on a new project, the first step is to create an empty git repository in your git hosting tool (Gitlab, Gitolite, Github...) and copy the repository URL.

If you don't have much code yet or don't mind moving it from where it currently resides, you may use the same clone command as before:

\begin{lstlisting}[style=BashInputStyle]
  $ git clone <url> [<directory>]
\end{lstlisting}

This time, you will get an information message \textit{You appear to have cloned empty repository}.

A good practice is to create a first commit with useful files like \texttt{README.txt} and \texttt{.gitignore}, adding them and committing: \marginnote{We will cover \texttt{git add} and \texttt{git commit} in more details later.}

\begin{lstlisting}[style=BashInputStyle]
  $ echo "Hello World" > README.txt
  $ touch .gitignore
  $ git add .
  $ git commit -m "Initial Commit"
  $ git push origin master
\end{lstlisting}

\subsection{Creating a new repository: existing code}

On the other hand, if you have just created an empty repository on your Git server and would like to add existing code to it, you can initialize a git repository in the top-level folder of your code in its current location: 

\marginnote{The \textbf{.gitignore} appearing here is a very important step when creating a repo; do not neglect it. It is much wiser to ignore files from the start then to clean it up later. See the following section on ignoring files.}
\begin{lstlisting}[style=BashInputStyle]
  $ cd /path/to/project
  $ git init
  $ git remote add origin <url>
  $ touch .gitignore
  $ # edit your .gitignore until you are happy with 'git status'
  $ git add .
  $ git commit -m "Initial Commit"
  $ git push origin master
\end{lstlisting} 

This will initialize a local git repository and define a remote called \texttt{origin}. The URL must be the one of the remote repository hosted on the Git server. The next commands add the entire existing code to your index and create a first commit contaning everything you have just added. 

In \marginnote{Although it has not been shown explicitly here, \texttt{master} is the default branch name after creating a git repository.} the last 2 scenarios, the \texttt{git push} command instructs to push the branch called \texttt{master} to the remote repository. 

\section{Exploring the Repository}

So we have just cloned or initialized our first repository, and have seen the difference between the \textbf{working tree} and \textbf{local repository}.
Most of the git commands we use will interact with the local git repository to write and read information to and from the working tree.
Before we actually change anything, let's take a look at the commands available to browse the current status of a project and its history.

\subsection{Listing Remotes}

The \texttt{git clone} command initiated a local repository replicating a \textbf{remote} repository, identified by a URL.
Information related to the state of \textbf{remote} repostories is stored in a dedicated area within the Git repository.
For a name-only list of remotes, we use:

\begin{lstlisting}[style=BashInputStyle]
  $ git remote
  origin
\end{lstlisting}

Every remote is identified by a short-hand name: \textbf{origin} in this case which is the default name Git assigns to a remote when it is cloned.
We can rename a remote repository to give it a more expressive name.
Let's rename it to \texttt{gitlab}, to remind us that the remote is hosted on our local Gitlab server.

\begin{lstlisting}[style=BashInputStyle]
  $ git remote rename origin gitlab
\end{lstlisting}


There is a verbose version of \texttt{git remote} which lists the URLs to each remote.
\marginnote{There are actually 2 URLs defined for each remote, allowing to decouple \texttt{fetch} from \texttt{push} to use different protocols or even different paths for some very specific configurations.}

\begin{lstlisting}[style=BashInputStyle]
  $ git remote -v
  gitlab  git@gitlab.server.com:login/lesson1 (fetch)
  gitlab  git@gitlab.server.com:login/lesson1 (push)
\end{lstlisting}

Without looking at an exhaustive list of remote-related commands, let's just mention that it is possible to add and delete a remote, as well as set a new URL for an existing remote.
For example, we can set an altenate but equivalent notation of the URL:

\marginnote{This is a 1-line command displayed in 2 lines for readability. I use the "\textbackslash" character and an indented second line to represent this.}

\begin{lstlisting}[style=BashInputStyle]
  $ git remote set-url gitlab \
     ssh://git@gitlab.server.com:login/lesson1.git
\end{lstlisting}

\subsection{Local and Remote branches}

As we will see throughout this course, \textbf{branches} are omnipresent in Git.
Therefore, commands which list branches and show their relationships are particularly useful.
The simplest way to list branches is:

\marginnote{Omitting the \texttt{-a} option of \texttt{git branch} will display only the local branches.}

\begin{lstlisting}[style=BashInputStyle]
  $ git branch -a
  * `master`
  §remotes/origin/HEAD -> origin/master§
  §remotes/origin/master§
\end{lstlisting}

The output lists local and remote branches.
The local branches are first displayed in default color, with the exeption of the \textbf{currently checked-out} branch, which is displayed in green and has an asterisk next to it.
Here, the current branch is master.

The remote branches are displayed in red and are prefixed with \texttt{remotes/alias/} for readability.
There's a special \texttt{HEAD} pointing on a remote branch which simply identifies the remote branch set as the project's default branch, which will be checked out locally when doing a \texttt{git clone}.

\subsection{Viewing the history}

To view the history of the current branch:

\begin{lstlisting}[style=BashInputStyle]
  $ git log
\end{lstlisting}

This will display a verbose output of the log in a scrollable text viewer.
\marginnote{The log manpage \texttt{git log --help} is a great reference as well as online resources such as \url{http://gitready.com/advanced/2009/01/20/bend-logs-to-your-will.html}
\\ \vspace{0.5cm}
\noindent The last command shows a textual graph of the history. A useful and complete git log graph is given at \url{http://stackoverflow.com/questions/1057564/pretty-git-branch-graphs}}
Git log has plenty of useful options, I will list a few of them here and let you experiment with them.

\begin{lstlisting}[style=BashInputStyle]
  $ git log -n 3
  $ git log --pretty=oneline
  $ git log --pretty=[short, medium, full, fuller, raw...]
  $ git log --pretty=format:'%h - %d %s (%cr) <%an>'
  $ git log --pretty=format:'%h -%C(red)%d%C(reset) %s (%cr) <%an>'
  $ git log --oneline
  $ git log --since "3 hours ago"
  $ git log --since "1 week ago"
  $ git log --graph --pretty=format:' ... '
\end{lstlisting}

\subsection{Graphical overview of branches}

We have seen that there are command-line ways of comparing commits in branches with textual graphs, but this can get complicated when dealing with multiple branches and remotes.
An alternative is to use a graphical user interface, such as Gitk which is distributed in msysgit and included in Linux and MacOS X installations.
To open Gitk:

\begin{marginfigure}%
  \centering
  \includegraphics[width=\linewidth]{gitk-start.png}
  \label{fig:gitk-start}
  \caption{Our very simple project displayed in Gitk. There is only 1 local branch with a linear history.}
\end{marginfigure}

\begin{lstlisting}[style=BashInputStyle]
  $ gitk --all
\end{lstlisting}

This is by far the fastest and easiest way to get an overview of the local and remote branches.
In Gitk, the local branches are green, the current local branch is green and bold, and remote branches are prefixed with a peach path.
Our project also has a yellow label, which is a tag.
Like a branch, a \texttt{tag} is little more than a label on a certain commit, the main difference being that a tag is designed to be permanent to label an important point in the project history for future reference (code release, development milestone...), whereas a branch is meant to evolve.
We'll get back to branches and tags later.

\subsection{Working tree status}

We are about to start applying changes to our project.
A useful command to track the status of the different files in Git is \texttt{git status}:

\begin{lstlisting}[style=BashInputStyle]
  $ git status
\end{lstlisting}

As we have not yet modified anything, our git status output is very short:

\begin{lstlisting}[style=BashInputStyle]
  On branch master
  nothing to commit (working directory clean)
\end{lstlisting}

Git status is more verbose when you start changing things.
It will group files in categories according to their state: \textbf{staged}, \textbf{modified} and \textbf{untracked}.

\section{Applying our first changes to master}
In many workflows, it is usually not recommended to work directly on the master branch.
We will do so only this first time to keep things simple.
In the future, we will always apply modifications in isolated branches, and bring these changes back onto master if desireable.

\subsection{Git add to stage an untracked file}

Let's create a new file and see the output of git status:

\begin{lstlisting}[style=BashInputStyle]
  $ echo hello > newfile.txt
  $ git status
  # On branch master
  # Untracked files:
  #   (use "git add <file>..." to include in what will be committed)
  #
    §newfile.txt§
  nothing added to commit but untracked files present (use "git add" to track)
\end{lstlisting}

Git status is more verbose, it tells us that there's untracked content inside the working tree.
All files added to a git working tree are untracked, and will stay untracked unless explicitly defined as tracked content in the repository.

\begin{marginfigure}%
  \centering
  \includegraphics[width=\linewidth]{gitadd-schema.pdf}
  \label{fig:gitadd}
  \caption{Git add on a file will stage all modifications in the file. It also adds untracked files to the staging area.}
\end{marginfigure}

\begin{lstlisting}[style=BashInputStyle]
  $ git add newfile.txt
  $ git status
    On branch master
    Changes to be committed:
      (use "git reset HEAD <file>..." to unstage)
  
    `new file:   newfile.txt`
\end{lstlisting}

\subsection{Our first commit}

The staging area is unique to Git, and makes it very powerful.
\marginnote{The term commit is also used in SVN, but it is very different here. An SVN commit is non-reversible and applies changes on the remote repository. A Git commit is a local operation and can easily be undone and edited before it reaches a remote server.}
Successive modifications can be applied until the developer is satisfied with the changes, and is ready to commit them.
Committing changes will define a new point in the project history with a description, and move the HEAD and current branch to the new commit.

\begin{marginfigure}%
  \centering
  \includegraphics[width=\linewidth]{gitcommit-schema.pdf}
  \label{fig:gitcommit}
  \caption{Git commit creates a new point in history, applying the changes in the staging area.}
\end{marginfigure}

\begin{lstlisting}[style=BashInputStyle]
  $ git commit -m "Added new file"
  [master 494822d] Added new file
   1 file changed, 1 insertion(+)
   create mode 100644 newfile.txt
\end{lstlisting}

Above, the commit message is given inline with the \texttt{-m} option, but it may be entered via the default text editor by omitting this option.

\subsection{Git add to stage modifications in tracked files}

Let's add more content to our new file and commit these changes.

\begin{lstlisting}[style=BashInputStyle]
  $ echo "hello again" >> newfile.txt
  $ git status
    On branch master
    Your branch is ahead of 'origin/master' by 1 commit.
  
    Changes not staged for commit:
      (use "git add <file>..." to update what will be committed)
      (use "git checkout -- <file>..." to discard changes in working directory)
   
    §modified:   newfile.txt§
   
no changes added to commit (use "git add" and/or "git commit -a")
\end{lstlisting}

Typing \texttt{git commit} at this point will not apply anything, because the modifications have not been added to the staging area.
It is possible to apply changes on a file basis in the same way as we have staged an untracked file previously:

\begin{lstlisting}[style=BashInputStyle]
  $ git add newfile.txt
  $ git status
    On branch master
    Your branch is ahead of 'origin/master' by 1 commit.
  
    Changes to be committed:
      (use "git reset HEAD <file>..." to unstage)
  
    `modified:   newfile.txt`
 
\end{lstlisting}

Committing now will apply the modifications.

\begin{marginfigure}%
  \centering
  \includegraphics[width=\linewidth]{gitk-2commits.png}
  \label{fig:gitk2commits}
  \caption{Gitk after 2 commits on the local master branch}
\end{marginfigure}

\begin{lstlisting}[style=BashInputStyle]
  $ git commit -m "Added a hello message to new file"
  [master 25d79d3] Added a hello message to new file
   1 file changed, 1 insertion(+)
\end{lstlisting}

\subsection{Git add -p: content-based staging}

Probably the cleanest way to stage changes in Git is to use the \texttt{-p} option of \texttt{git add}:

\begin{lstlisting}[style=BashInputStyle]
  $ git add -p
\end{lstlisting}

This doesn't require any files as argument.
\texttt{git add -p} will initiate an interactive session during which each modification will be displayed, and the user must decide whether to add the changes to the staging area, or leave it as modified. \marginnote{\texttt{git add -p} common commands: \textbf{y}: stage, \textbf{n}: do not stage, \textbf{d}: don't stage and skip rest of file, \textbf{s}: split into smaller chunks and ask again.}
The great thing about this command is that the modifications are presented per line (or groups of adjacent lines) rather than by file, meaning that you can stage a part of a file without staging the rest of it.
That way, it's very easy to isolate features in different commits, even if these involve modification in the same set of files.
The interactive prompt will end when all changes have been cycled through.

Suppose I have made two modifications in \texttt{calc.py} and that I have selected \textbf{y} for only one of these when running \texttt{git add -p}.
The file with contain both staged and modified lines:

\begin{lstlisting}[style=BashInputStyle]
  $ git status
    On branch master
    Your branch is ahead of 'origin/master' by 2 commits.
  
    Changes to be committed:
      (use "git reset HEAD <file>..." to unstage)
  
    `modified:   python/calc.py`
  
    Changes not staged for commit:
      (use "git add <file>..." to update what will be committed)
      (use "git checkout -- <file>..." to discard changes in worki...
  
    §modified:   python/calc.py§
\end{lstlisting}

Git commit at this point will apply the staged code to the commit, and leave the file in the modified state with the remaning, unstage lines.
We have already covered commits, so let's stay in this state to explore the \textbf{diff} command.

\marginnote{\texttt{git diff} will display the full diff between the modified content and HEAD. Already staged data does not appear.}
\begin{lstlisting}[style=BashInputStyle]
  $ git diff
  diff --git a/python/calc.py b/python/calc.py
  index b66135e..9178e52 100644
  --- a/python/calc.py
  +++ b/python/calc.py
  @@ -17,6 +17,7 @@ def mul(op1, op2):
   def div(op1, op2):
     return op1 / op2
 
  `+ New comment, I will not stage it for now`
   def pow(op1, op2):
     return op1 ** op2
 \end{lstlisting}
\marginnote{\texttt{git diff} with the \textbf{cached} option is the opposite: it displays the diff between the staged content and HEAD, and ignores modified content.}
 \begin{lstlisting}[style=BashInputStyle]
  $ git diff --cached
  diff --git a/python/calc.py b/python/calc.py
  index c27b82f..b66135e 100644
  --- a/python/calc.py
  +++ b/python/calc.py
  @@ -2,6 +2,7 @@ import os
   import sys
   import argparse
 
  `+ New comment. I will stage this first`
   choices_cmd = ['add', 'sub', 'mul', 'div', 'pow', 'mod', 'floor']
 
   def add(op1, op2):
\end{lstlisting}

\subsection{Stage all modified and untracked in 1 command}

I recommend avoiding this as much as possible, but for the record, it is possible to use a single command to add all untracked and modified files to the staging area:

\begin{lstlisting}[style=BashInputStyle]
  $ git add -A
\end{lstlisting}

\subsection{Commit all modified files in 1 command}

Another shortcut worth knowing but which should be avoided in most cases is the commit option to stage all outstanding modifications and record a new commit:
\begin{marginfigure}%
  \centering
  \includegraphics[width=\linewidth]{gitcommit-am-schema.pdf}
  \label{fig:gitcommit-am}
  \caption{Git commit -am stages all modified files and applies a commit immediately after.}
\end{marginfigure}
\begin{lstlisting}[style=BashInputStyle]
  $ git commit -am "One big commit"
\end{lstlisting}

I prefer avoiding this because commits should be decoupled by features and it is rare that there are no superfluous modifications when developing a feature.
This \texttt{-a} option doesn't leave time to verify the contents of the commit before applying changes, whereas decoupling staging and commit phases allows to view the staged changes with commands like:

\begin{lstlisting}[style=BashInputStyle]
  $ git diff --cached
\end{lstlisting}

\subsection{Unstaging stuff}

To unstage work, \texttt{git reset} comes in handy.
This command is very multi-purpose in Git, for now we will use its default behavior which is to unstage modifications without deleting them.
\texttt{git reset} without arguments will unstage all currently staged changes, and bring them back to their former state (modified or untracked).

\marginnote{Although it's technically possible to combine \texttt{[tree-ish]} and \texttt{[file]} options to git reset, this will not actually delete the commit if there are other non-reset changes in the commits. Thus the content is reset as desired, but applied this reset requires to stage and commit the modifications in new commits. Git revert has a similar behavior in the sense that we are adding commits to actually undo work.}

\begin{lstlisting}[style=BashInputStyle]
  $ git reset 
\end{lstlisting}

This is only a partial explanation of why git reset does this.
Like many git commands which accept a \textbf{tree-ish} as optional argument, git reset assumes \texttt{HEAD} when unspecified.
The general \texttt{git reset} actually undoes commits from the current state up until (and not including) the commit referenced by the tree-ish.

\begin{lstlisting}[style=BashInputStyle]
  $ git reset [tree-ish]
\end{lstlisting}

Finally, unstaging can also be applied on a per-file basis.

\begin{lstlisting}[style=BashInputStyle]
  $ git reset [file]
\end{lstlisting}

\section{Untracking and Deleting files}

Deleting a file is treated in the same way as other modifications in Git.
There are two ways to remove a file from version control.
The \textit{clean and recommended} way consists in invoking the file/folder removal directly with Git.
Removing content directly from the filesystem is not as clean, but is \textbf{easily recoverable} in Git.
\marginnote{SVN users: yes, that my sound like magic.}
We will have a look at both methods in this section.

\subsection{Removing files using Git}

The cleanest way to remove a file from disk as well as from the Git repository is to use \texttt{git rm} on it:

\begin{lstlisting}[style=BashInputStyle]
  $ git rm <file>
\end{lstlisting}

This performs two things: it deletes the file from disk, and \textbf{stages} the removal of this file.
The next commit will thus include the deletion of the file.

If you want to untrack a file from Git \textit{without deleting it} locally, use the \texttt{cached} option:

\begin{lstlisting}[style=BashInputStyle]
  $ git rm --cached <file>
\end{lstlisting}

\marginnote{rm cached comes in very handy to untrack things you forgot to add to your .gitignore file}
Git will stage the file to be removed in the next commit, but will not touch the file locally.

\subsection{Removing files outside of Git}

So, what happens when you delete some content locally that Git has been versioning?
Git will detect the missing content, as a \texttt{git status} will tell you:

\begin{lstlisting}[style=BashInputStyle]
  $ rm README.md
  $ git status
  On branch master
  Your branch is up-to-date with 'origin/master'.

  Changes not staged for commit:
    (use "git add/rm <file>..." to update what will be committed)
    (use "git checkout -- <file>..." to discard changes in working directory)

      §deleted:    README.md§

  no changes added to commit (use "git add" and/or "git commit -a")
\end{lstlisting}

\marginnote{as opposed to a git rm README.md, which stages the deletion.}
The deleted content is equivalent to a modification: it is not staged by default.
It is possible to execute \textit{git rm} on each deleted file, even though it's not on disk anymore, which will tell git to stage the deletion.
However, this can be tedious if you've removed lots of files.
Instead, you can ask git to synchronize the staging area with all modified content with:

\begin{lstlisting}[style=BashInputStyle]
  $ git add -u
\end{lstlisting}

In case of deleted files, \texttt{git add -u} will stage every modification at once.
The next commit will apply the file removals.

\pagebreak

\section{Pushing work to a remote}

We have progressed locally on the master branch.
So far, all our modifications are local and we have not shared our work with anyone.

\begin{marginfigure}%
  \centering
  \includegraphics[width=\linewidth]{gitcommit-pre-push.png}
  \label{fig:gitcommit-pre-push}
  \caption{Local and remote repo status before the push. Local master is 4 commits ahead of origin's.}
\end{marginfigure}
\begin{marginfigure}%
  \centering
  \includegraphics[width=\linewidth]{gitpush-schema.pdf}
  \label{fig:gitpush-schema}
\end{marginfigure}
\begin{marginfigure}%
  \centering
  \includegraphics[width=\linewidth]{gitcommit-post-push.png}
  \label{fig:gitcommit-post-push}
  \caption{After pushing master to origin, both branches are at the same point in history.}
\end{marginfigure}

The \texttt{git push} command is used to push individual branches from the local repository to a remote one.
The command produces the following output.

\begin{lstlisting}[style=BashInputStyle]
  $ git push origin master 
  Counting objects: 17, done.
  Delta compression using up to 8 threads.
  Compressing objects: 100% (10/10), done.
  Writing objects: 100% (14/14), 1.29 KiB, done.
  Total 14 (delta 4), reused 0 (delta 0)
  To git@gitlab.server.com:login/lesson1
     71aaf45..37b9258  master -> master
\end{lstlisting}

Git push is only possible if the remote branch to which we are pushing has not been updated in the meantime.
In other words, it succeeds \textit{if the remote branch is in the direct history of the local branch}.

If pushing is not possible because the remote branch has been updated (usually, by someone else), we enter a multi-user workflow which we will cover later.
\marginnote{I use integrate here to stay general. There are many ways to do this, like merge, rebase, hard reset...}
In a few words, the remote modifications must first be \textbf{fetched} locally and "integrated" in our work before we can contribute.

\section{Applying our changes to a feature branch first}

So far we have worked on a master branch.
We will conclude this first session by working in feature branches, and merging them into the master branch once they are stable.

\subsection{Branching: very lightweight}

Branching in Git is as simple as adding a new label to an existing commit.
\marginnote{This is very different from some other SCMs, where creating a branch is a lengthy process involving copying the entire working tree.}
The working tree is left entirely intact when a branch is created.
\begin{marginfigure}%
  \centering
  \includegraphics[width=\linewidth]{gitbranch-feature.png}
  \label{fig:gitbranch-feature}
  \caption{A new branch is just a new label.}
\end{marginfigure}
\begin{lstlisting}[style=BashInputStyle]
  $ git branch -a
  * `master`
    §remotes/origin/HEAD -> origin/master§
    §remotes/origin/master§

    $ git branch feature

  $ git branch -a
    feature
  * `master`
    §remotes/origin/HEAD -> origin/master§
    §remotes/origin/master§
\end{lstlisting}

We have a new local branch called feature branch, currently pointing to the same commit as master.
We are not quite ready to start applying changes, because \texttt{git branch} and \texttt{git status} tell us that we are still in master.
To progress in the \texttt{feature} branch rather then \texttt{master} we must use \textbf{checkout}:

\begin{lstlisting}[style=BashInputStyle]
  $ git checkout feature
  $ git branch -a
  * feature
    master
    remotes/origin/HEAD -> origin/master
    remotes/origin/master
\end{lstlisting}

\subsection{Integrating a feature branch back into master}

Now, we can start adding commits to the feature branch just like we have done for master.
Once we are satisfied with our work, we can merge it back into master.
We will not go into too much details for now, but the general idea is:

\begin{marginfigure}%
  \centering
  \includegraphics[width=\linewidth]{gitmerge-ff.pdf}
  \label{fig:gitmerge-ff}
  \caption{Merging in fast-forward mode.}
\end{marginfigure}

\begin{lstlisting}[style=BashInputStyle]
  $ git checkout master
  $ git merge feature
\end{lstlisting}

If master has not progressed since we branched off with feature, the master branch will be \textbf{fast-forwarded} to the feature branch.
If a branching history is preferred, we can force it with \texttt{no-ff}:

\begin{marginfigure}%
  \centering
  \includegraphics[width=\linewidth]{gitmerge-noff.pdf}
  \label{fig:gitmerge-noff}
  \caption{Merging in no-fast-forward mode creates an explicit merge commit.}
\end{marginfigure}

\begin{lstlisting}[style=BashInputStyle]
  $ git checkout master
  $ git merge feature --no-ff
\end{lstlisting}

This creates an additional commit, called a merge commit, with 2 parents: the last commit on each branch prior to the merge.
If master \textit{did} progress since we branched off with the feature branch, the merge operation will automatically result in an extra merge commit.

In the next session, we will dig further into workflows dealing with multiple branches, multiple people, and learn other ways to integrate changes such as \textbf{rebase}.

\bibliography{../common/refs}
\bibliographystyle{plainnat}

\end{document}


% \marginnote{}